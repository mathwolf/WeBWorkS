## DESCRIPTION
## College Algebra, linear rates of change
## ENDDESCRIPTION


## DBsubject(Algebra)
## DBchapter()
## DBsection()
## Date(06/17/2016)
## Institution(Colorado Community College System)
## Author(Steven Wray)
## MO(1)
## KEYWORDS('college algebra', 'linear equations')


###########################
#  Initialization

DOCUMENT();

loadMacros(
"PGstandard.pl",
"MathObjects.pl",
"AnswerFormatHelp.pl",
"PGML.pl",
"PGcourse.pl",
"scaffold.pl",
);

TEXT(beginproblem());
$showPartialCorrectAnswers = 1;

###########################

#  Setup 
$u=non_zero_random(1,3);  # three possibilities for the details of the rental car contract
if ($u == 1) {  
  $m = 0.15;
  $b = 65;
} elsif ($u == 2) {
  $m = 0.25;
  $b = 55;
} else {
  $m = 0.35;
  $b = 45;
}

# Parameters for solution
$mileage = non_zero_random(4,7);
$mileage = 10 * $mileage;
$ans1 = $m * $mileage + $b;

$ans2 = $mileage + 13;
$cost = $m * $ans2 + $b;

$cost2 = non_zero_random(1,4);
$cost2 = 80 + 10 * $cost2;
$ans3 = round(($cost2 - $b) / $m);

###########################
#  Main text

Scaffold::Begin();
    
Section::Begin("Part 1");

BEGIN_PGML
Suppose that the cost of a rental car is $[`{[$b]}`] for a week plus $[`{[$m]}`] per mile driven.  An equation that represents the total weekly cost is

[`` y = {[$m]} x + {[$b]}, ``]

where [`x`] is the number of miles driven.  What is the total cost of the car if you
travel [`{[$mileage]}`] miles?

 [_______________]{$ans1} [@ AnswerFormatHelp("numbers") @]*


END_PGML


Section::End();
    
Section::Begin("Part 2");

BEGIN_PGML
If the total cost of your rental for one week is $[`{[$cost]}`], how many miles did you travel? 

 [_______________]{$ans2} [@ AnswerFormatHelp("numbers") @]*

END_PGML

Section::End();

Section::Begin("Part 3");

BEGIN_PGML
Suppose that you have a maximum of $[`{[$cost2]}`] to spend on a car rental.  What is the maximum number of miles that you can travel?  Round your answer to the nearest whole number.

 [_______________]{$ans3} [@ AnswerFormatHelp("numbers") @]*

END_PGML

Section::End();
  
Scaffold::End();

############################
#  Solution

#BEGIN_PGML_SOLUTION
#Solution explanation goes here.
#END_PGML_SOLUTION

COMMENT('MathObject version. Uses PGML.');

ENDDOCUMENT();