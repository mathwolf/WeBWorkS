## DESCRIPTION
## College Algebra, slopes of lines
## ENDDESCRIPTION


## DBsubject(Algebra)
## DBchapter()
## DBsection()
## Date(06/17/2016)
## Institution(Colorado Community College System)
## Author(Steven Wray)
## MO(1)
## KEYWORDS('college algebra', 'polynomial equations','substitution')


###########################
#  Initialization

DOCUMENT();

loadMacros(
"PGstandard.pl",
"MathObjects.pl",
"AnswerFormatHelp.pl",
"PGML.pl",
"PGcourse.pl",
"parserRadioButtons.pl",
);

TEXT(beginproblem());
$showPartialCorrectAnswers = 1;

###########################
#  Setup


$a=non_zero_random(1,3); # choose parallel, perpendicular, or neither as answer

# used to generate LHS coefficients
$b=non_zero_random(2,3);
$c=non_zero_random(4,6);
$d=non_zero_random(2,5);

# RHS coefficients
$e=non_zero_random(3,7);
$f=non_zero_random(8,13);

###########################
#  Main text
#  There are three cases for the three possible relationships between the lines.
#  Use radio buttons for answer

if ($a == 1) {               # parallel lines
  $radio = RadioButtons(
    ["parallel","perpendicular","neither"],
    "parallel" # correct answer
  );

BEGIN_PGML

Describe the following pair of lines as parallel, perpendicular, or neither.  

[`` {[$c]}x - {[$b]}y = {[$e]} \\ {[$b * $d]}y - {[$c * $d]}x = {[$f]} ``]

[_______________]{$radio} 

END_PGML

} elsif ($a ==2) {               # perpendicular lines
  $radio = RadioButtons(
    ["parallel","perpendicular","neither"],
    "perpendicular" # correct answer
  );

BEGIN_PGML

Describe the following pair of lines as parallel, perpendicular, or neither.  

[`` {[$c]}x - {[$b]}y = {[$e]} \\ {[$c * $d]}y + {[$b * $d]}x = {[$f]} ``]

[_______________]{$radio} 



END_PGML

} else {               # neither parallel nor perpendicular
  $radio = RadioButtons(
    ["parallel","perpendicular","neither"],
    "neither" # correct answer
  );

BEGIN_PGML

Describe the following pair of lines as parallel, perpendicular, or neither.  

[`` {[$c]}x - {[$e]} =  {[$b]}y \\ {[$b * $d]}y + {[$c * $d]}x = {[$f]} ``]

[_______________]{$radio} 

END_PGML

}


############################
#  Solution

#BEGIN_PGML_SOLUTION
#Solution explanation goes here.
#END_PGML_SOLUTION

COMMENT('MathObject version. Uses PGML.');

ENDDOCUMENT();